\documentclass[UTF8]{article}
    \usepackage{CTEX}
    \usepackage{color}
    \usepackage{geometry}
    \geometry{left=2.0cm,right=2.0cm,top=2.5cm,bottom=2.5cm}
    \author{Cangye@geophyx.com}
	\date{}
    \title{AI 人工智能工程师直通车}
\begin{document}
    \maketitle
    \tableofcontents

	\section{数学基础}
		\subsection{空间、张量与函数基础}
对于物理场多场线性约束方程可以表示成如下的形式:
\begin{equation}
\mathcal{L}\cdot {v}(x,y,z)+\mathcal{M}\cdot {f}=0
\end{equation} 
其中 代表二阶线性偏微分算子$\nabla\nabla,\nabla\times\nabla\times$等
		\subsection{概率分析基础}
对于物理场多场线性约束方程可以表示成如下的形式:
\begin{equation}
\mathcal{L}\cdot {v}(x,y,z)+\mathcal{M}\cdot {f}=0
\end{equation} 
其中 代表二阶线性偏微分算子$\nabla\nabla,\nabla\times\nabla\times$等
		\subsection{空间、张量进阶-矩阵与优化问题}
对于物理场多场线性约束方程可以表示成如下的形式:
\begin{equation}
\mathcal{L}\cdot {v}(x,y,z)+\mathcal{M}\cdot {f}=0
\end{equation} 
其中 代表二阶线性偏微分算子$\nabla\nabla,\nabla\times\nabla\times$等
		\subsection{函数进阶-动态系统}
对于物理场多场线性约束方程可以表示成如下的形式:
\begin{equation}
\mathcal{L}\cdot {v}(x,y,z)+\mathcal{M}\cdot {f}=0
\end{equation} 
其中 代表二阶线性偏微分算子$\nabla\nabla,\nabla\times\nabla\times$等
                \subsection{概率分析进阶-随机类方法}
对于物理场多场线性约束方程可以表示成如下的形式:
\begin{equation}
\mathcal{L}\cdot {v}(x,y,z)+\mathcal{M}\cdot {f}=0
\end{equation} 
其中 代表二阶线性偏微分算子$\nabla\nabla,\nabla\times\nabla\times$等
	\section{基础算法的数学原理与实践}
		\subsection{$y=f(\vec{x})$类问题与实践}
			\subsubsection{单层与多层神经网络}
对于物理场多场线性约束方程可以表示成如下的形式:
\begin{equation}
\mathcal{L}\cdot {v}(x,y,z)+\mathcal{M}\cdot {f}=0
\end{equation} 
其中 代表二阶线性偏微分算子$\nabla\nabla,\nabla\times\nabla\times$等
			\subsubsection{SVM与RBF网络}
对于物理场多场线性约束方程可以表示成如下的形式:
\begin{equation}
\mathcal{L}\cdot {v}(x,y,z)+\mathcal{M}\cdot {f}=0
\end{equation} 
其中 代表二阶线性偏微分算子$\nabla\nabla,\nabla\times\nabla\times$等
			\subsubsection{SOM与PCA分析}
对于物理场多场线性约束方程可以表示成如下的形式:
\begin{equation}
\mathcal{L}\cdot {v}(x,y,z)+\mathcal{M}\cdot {f}=0
\end{equation} 
其中 代表二阶线性偏微分算子$\nabla\nabla,\nabla\times\nabla\times$等
			\subsubsection{正则化问题}
\input{2.1.4.tex}
		\subsection{$y=f(\vec{x};t)$类问题与实践}
                        \subsubsection{动态规划与马尔科夫决策}
对于物理场多场线性约束方程可以表示成如下的形式:
\begin{equation}
\mathcal{L}\cdot {v}(x,y,z)+\mathcal{M}\cdot {f}=0
\end{equation} 
其中 代表二阶线性偏微分算子$\nabla\nabla,\nabla\times\nabla\times$等
                        \subsubsection{混沌吸引子与RNN、LSTM}
对于物理场多场线性约束方程可以表示成如下的形式:
\begin{equation}
\mathcal{L}\cdot {v}(x,y,z)+\mathcal{M}\cdot {f}=0
\end{equation} 
其中 代表二阶线性偏微分算子$\nabla\nabla,\nabla\times\nabla\times$等
                \subsection{信息论与概率类问题}
                        \subsubsection{EM算法}
对于物理场多场线性约束方程可以表示成如下的形式:
\begin{equation}
\mathcal{L}\cdot {v}(x,y,z)+\mathcal{M}\cdot {f}=0
\end{equation} 
其中 代表二阶线性偏微分算子$\nabla\nabla,\nabla\times\nabla\times$等
                        \subsubsection{贝叶斯网络与主题模型}
对于物理场多场线性约束方程可以表示成如下的形式:
\begin{equation}
\mathcal{L}\cdot {v}(x,y,z)+\mathcal{M}\cdot {f}=0
\end{equation} 
其中 代表二阶线性偏微分算子$\nabla\nabla,\nabla\times\nabla\times$等
                        \subsubsection{决策树与随机森林}
\input{2.4.3.tex}
                        \subsubsection{PCA与ICA}
\input{2.5.4.tex}
    \section{机器学习库推荐与使用}
        \subsection{Java,Python与R相关库(1)}
对于物理场多场线性约束方程可以表示成如下的形式:
\begin{equation}
\mathcal{L}\cdot {v}(x,y,z)+\mathcal{M}\cdot {f}=0
\end{equation} 
其中 代表二阶线性偏微分算子$\nabla\nabla,\nabla\times\nabla\times$等
        \subsection{Java,Python与R相关库(2)}
对于物理场多场线性约束方程可以表示成如下的形式:
\begin{equation}
\mathcal{L}\cdot {v}(x,y,z)+\mathcal{M}\cdot {f}=0
\end{equation} 
其中 代表二阶线性偏微分算子$\nabla\nabla,\nabla\times\nabla\times$等
        \subsection{Caffe,Tensorflow与Mlib(1)}
\input{3.3.tex}
        \subsection{Caffe,Tensorflow与Mlib(2)}
\input{3.4.tex}
        \subsection{Caffe,Tensorflow与Mlib(3)}
\input{3.5.tex}
    \section{机器学习实战}
        \subsection{波形分析(NN)}
\input{4.1.tex}
        \subsection{主成分分析(PCI+ICA)}
\input{4.2.tex}
        \subsection{人脸识别(CNN)}
\input{4.3.tex}
        \subsection{图像处理(DNN+CNN)}
\input{4.4.tex}
        \subsection{自然语言处理(RNN+LSTM)}
\input{4.5.tex}
        \subsection{机器翻译(LSTM)}
\input{4.6.tex}
        \subsection{Fancy可视化}
\input{4.7.tex}      
    \section{面试攻略}
\input{5.tex}
\end{document}